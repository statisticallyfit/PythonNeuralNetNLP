\usepackage{fancyvrb,newverbs,xcolor} % for code highlighting



% OLD: 
% \usepackage[top=2cm, bottom=1.5cm, left=2cm, right=2cm]{geometry} % for page margins

% NEW Margin setting for guessing larger code cell outputs (a2 paper is wider).
\usepackage[left=2cm, % left margin
right=2cm,% right margin
top=3cm, % top margin
bottom=3cm,% bottom margin
a2paper% other options: a0paper, a1paper, a2paper, a3paper, a4paper, a5paper, a6paper, and many more.
]{geometry}

\usepackage[english]{babel}
% Ana: adding graphics package for images
\usepackage{graphics}
\usepackage{graphicx}

% change background color for inline code in
% markdown files. The following code does not work well for
% long text as the text will exceed the page boundary
%\definecolor{bgcolor}{HTML}{E0E0E0}
%\let\oldtexttt\texttt

% \renewcommand{\texttt}[1]{
% \colorbox{bgcolor}{\oldtexttt{#1}}
% }


%% Setting pythong ??? -----------------------------------------------------
%default_block_language: "lexer"
%default_inline_language: "lexer"


%% color and other settings for hyperref package -----------------------------
\hypersetup{
    bookmarksopen=true,
    linkcolor=blue,
    filecolor=magenta,
    urlcolor=RoyalBlue,
}

% Font Setup  ---------------------------------------------------------
\usepackage{unicode-math} % load 'fontspec' automatically
%\setmainfont{GFS Artemisia}
\setmainfont{Libertinus Sans} 
%\setmainfont{Palatino}
%\setmainfont{Alegreya}
\setmathfont{TeX Gyre Schola Math}
%\setmathfont{Junicode}


% Code syntax highlighting ---------------------------------------------------

% OLD PART -----------------
%\usepackage{minted}
%\usemintedstyle{manni}
\setmonofont{Inconsolata}
%\setmonofont{Consolas}
% ---------------------------


% Preliminary macro things for code (snatched from macros in REPORT):  ------
\newcommand\CodeFontSizeSmall{\fontsize{9pt}{9pt}\selectfont}

\definecolor{originalmannibg}{HTML}{f2f2ff}
\colorlet{BasePurple}{originalmannibg!90}
\newcommand{\lighten}[3]{% Reference Color, Percentage, New Color Name
    \colorlet{#3}{#1!#2!white}
}
\lighten{BasePurple}{50}{mannibg}

% Code things --------------------
\usepackage{minted}
\usepackage{verbatim}  % has commenting



\usemintedstyle{manni}

%\setmonofont{Inconsolata} % setting code font
\setmonofont{Fira Mono}

% General code environment, used like: \begin{code}{python} .... \end{code}
% NOTE: this is how to nest two environments together: 
%\newenvironment{code}[2][]
% {\vspace{-3pt}%
% \VerbatimEnvironment
%  \begin{adjustwidth}{30pt}{30pt}
%  \begin{minted}[
%    fontsize=\CodeFontSizeSmall,
%    breaklines, mathescape,
%    style=manni, bgcolor=mannibg,  #1]{#2}}
% {\end{minted}\end{adjustwidth} 
%     \vspace{-10pt}
% }
 
% TODO: test if possible to do \renewenvironment to renew the minted environment and just include this logic below whenever calling \begin{minted}[]{python} ... 
 
% Python code environment, used like \begin{pythonCode} ... \end{pythonCode}
\newenvironment{pythonCode}
 {\vspace{-3pt}%
 \VerbatimEnvironment
  \begin{adjustwidth}{30pt}{30pt}
  \begin{minted}[
    fontsize=\CodeFontSizeSmall,
    breaklines, mathescape,
    style=manni, bgcolor=mannibg]{python}}
 {\end{minted}\end{adjustwidth} 
     \vspace{-10pt}
 }



% General code output environment
\newenvironment{outputCode}
 {\VerbatimEnvironment
  \begin{adjustwidth}{30pt}{30pt}
  \begin{minted}[
    fontsize=\CodeFontSizeSmall,
    breaklines]{text}}
 {\end{minted}\end{adjustwidth}}


% Creating inline code font (equivalent to backticks in jupyter notebooks)
% Must use like: \pythoninline{...text here ... }
\newmintinline{python}{python3, fontsize=\CodeFontSizeSmall, bgcolor=mannibg}

%\newenvironment{mintInline}[1][]{\mintinline{latex}{#1}}{}
%\DeclareTextFontCommand{\mint}{\mintInline}


